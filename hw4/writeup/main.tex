\documentclass[10pt,drafclsnofoot,onecolumn]{IEEEtran}
\usepackage{graphicx}
\usepackage{anysize}
\usepackage{color}
\usepackage{balance}
\usepackage{enumitem}
\usepackage{listings}
\usepackage{cite}
\usepackage{hyperref}
\usepackage{url}      
\usepackage[margin=0.75in]{geometry}
\usepackage{amsmath}
\usepackage{array}
\marginsize{.75in}{.75in}{.75in}{.75in}
             
\begin{document}               
\begin{titlepage}
\title{\huge{Project 3:} \\ \large{The Kernel Crypto API}}
\author{
	Class: CS 444 \\ Professor Kevin McGrath
}

\null  % Empty line
\nointerlineskip  % No skip for prev line
\vfill
\let\snewpage \newpage
\let\newpage \relax
\maketitle
\begin{center}
\huge{Encrypted Block Device}\par
\vspace{2mm}
\large{Group 21 Members:}\par
\normalsize{David Baugh}\par
\normalsize{Justin Sherburne}\par
\vspace{8mm}
\large{\textbf{Abstract:}}\par 
\vspace{2mm}
\normalsize{This project's design was to implement a encrypted RAM Disk block device driver using the Linux Crypto API.}
\end{center}
\let \newpage \snewpage
\vfill 
\break % page break
\end{titlepage}

\section{Implementation steps}
\begin{itemize}
\item Source the correct enviroment: source /scratch/files/environment-setup-i586-poky-linux
\item Go into the menuconfig and set sbd to a Module:
	\begin{itemize}
	\item make menuconfig
	\item Go to device drivers
	\item Enter block drivers
	\item Go to SBD block driver
	\item Press M
	\item Save and exit
	\end{itemize}
\item Make the VM: make -j4 modules
\item Make the VM: make -j4 all
\item Start the vm: qemu-system-i386 -nographic -kernel arch/x86/boot/bzImage -drive file=core-image-lsb-sdk-qemux86.ext4 -enable-kvm -usb -localtime --no-reboot --append "root=/dev/hda rw console=ttyS0 debug"
\item Use the root account with no password
\item Go to the base directory: cd ../../
\item Copy the module into the base directory using scp from the drivers/block/ directory: 
	\begin{itemize}
	\item scp baughd@os2.engr.oregonstate.edu:/scratch/fall2017/21/linux-yocto-3.19.2/linux\_bak/drivers/block/sbd.ko sbd.ko
	\end{itemize}
\item Start up the module: insmod sbd.ko
\item Create the ext2 file directory: mkfs.ext2 /dev/sbd0
\item Create a directory to mount to: mkdir /mnt
\item Mount it: mount /dev/sbd0 /mnt
\item Enter some text into a random file on the mounted dir: echo "Testing 123 Can you find me" $>$ /mnt/test
\item Try to find that text on the device: grep -a "Testing" /dev/sbd0
\end{itemize}

\section{Questions}
\begin{enumerate}
\item{What do you think the main point of this assignment is?}
\\	\indent The main point of this assignment was to continue learning about kernel development and for this assignment, specifically, we were to dip our toes into the world of cryptology and drivers within the Linux kernel environment. \\

\item{How did you personally approach the problem? Design decisions, algorithm, etc.}
\\	\indent When we started this project the first hurdle was the to create the block device. During our research from the LDD3 link provide in the assignment page to other pages we found what itself calls a sample, extra-simple block driver created by Pat Patterson. We kept finding this block driver around and decided to base our project off of it. After getting it running as a module we moved on to the cryptology which to a bit longer to research and learn about but the actual crypto.h documentation was very helpful. \\

\item{How did you ensure your solution was correct? Testing details, for instance.}
\\ \begin{itemize}
	\item Try to find that text on the device: grep -a "Testing" /dev/sbd0
	\item After running the grep command, nothing should be returned if the encryption is successful. With no encryption, the text would have been found via grep. \\
	\end{itemize}

\item{What did you learn?}
\\ \indent This project was a study in debugging in a tight spot. After the module was in a implemented and started crashing we had to figure out how to read it's crash log and then research what it meant. Learning how to even implement the module was interesting.\\

\end{enumerate}

\section{Version Control Log}
\' \\ \centering
\begin{tabular}{c c c}
	\textbf{Detail} & \textbf{Author} & \textbf{Description}\\\hline
	
	\href{dirname "$origin"/basename "$origin" .git/commit/d8421e4c1c9dafedca6c0f887141d91d97e55b2e}{d8421e4} & David Baugh & added sbd.c, kconfig and Makefile editted but not added\\\hline
	\href{dirname "$origin"/basename "$origin" .git/commit/52a045e97dbb38253b5fb74100bfb2defd5a6b9b}{52a045e} & David Baugh & finished project\\\hline
	\href{dirname "$origin"/basename "$origin" .git/commit/7584bd1b23f60e808e3d8d5b16d95ffda4c4a55d}{7584bd1} & David Baugh & removed old sbd.c\\\hline
	\href{dirname "$origin"/basename "$origin" .git/commit/69b977a0b9cc36d22bcff524e5523b11a9c4421b}{69b977a} & Justin Sherburne & Crypto Kernel project push\\\hline
	\href{dirname "$origin"/basename "$origin" .git/commit/6939e21b15109231b66433b6a0b4fe5ab16dd54b}{6939e21} & Justin Sherburne & Crypto Kernel project push\\\hline
\end{tabular}

\section{Work Log}

\begin{itemize}
\item{Thursday, November 9th, 2017:}

\textsl{On Thursday we began researching device drivers, and simple implementations of what we needed to accomplish. Our research took us to two main options, one called Sbull.c, and one called sbd.c. We ended up using sbd (simple block driver) because it was exactly what we needed. }

\item{Friday, November 10th, 2017:}

\textsl{The day was spent frantically trying to get our OS2 paper written for the Linux / Windows/ FreeBSD comparison paper.}


\item{Sunday, November 12th, 2017:}

\textsl{Sunday was spent implementing the core of the sbd.c module. First we got the sbd.c module configured into the kernel, compiled, and ran the driver. Then we began researching crypto implementations for block drivers. We settled on an AES block encryption scheme, and files that need to be encrypted must be mounted to the driver we are creating.}

\item{Monday, November 13th, 2017:}

\textsl{Monday we implemented the Crypto functionality within the simple block driver. After implementing crypto, we built and began troubleshooting the driver. There was an issue where a null pointer was being passed into a function causing our driver to crash and hang the entire system, which was difficult to troubleshoot. But we got it working and running successfully.}

\item{Tuesday, November 14th, 2017:}

\textsl{Tuesday was spent creating the write-up and commiting code to Github (which we forgot to do...). Additionally, we created the patchfile to be submitted alongside our writeup.}

\end{itemize}

\end{document}
