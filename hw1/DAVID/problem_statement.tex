%% The following template is sourced from Kevin McGrath on 6 Oct. 2017 located:
%%http://web.engr.oregonstate.edu/cgi-bin/cgiwrap/dmcgrath/classes/17F/cs444/index.cgi?examples=examples/template.tex

\documentclass[letterpaper,10pt,titlepage]{article}

\usepackage{graphicx}                                        
\usepackage{amssymb}                                         
\usepackage{amsmath}                                         
\usepackage{amsthm}                                          

\usepackage{alltt}                                           
\usepackage{float}
\usepackage{color}
\usepackage{url}

\usepackage{balance}
\usepackage{enumitem}
\usepackage{pstricks, pst-node}

\usepackage{geometry}
\geometry{textheight=8.5in, textwidth=6in}

%random comment

\newcommand{\cred}[1]{{\color{red}#1}}
\newcommand{\cblue}[1]{{\color{blue}#1}}

\usepackage{hyperref}
\usepackage{geometry}

\def\name{Justin Sherburne}

%pull in the necessary preamble matter for pygments output
\input{pygments.tex}

%% The following metadata will show up in the PDF properties
\hypersetup{
  colorlinks = true,
  urlcolor = blue,
  pdfauthor = {\name},
  pdfkeywords = {cs460 ``computer science'' capstone problem statement},
  pdftitle = {AHS Competition Problem Statement},
  pdfsubject = {Problem Statement},
  pdfpagemode = UseNone
}

\begin{document}
\begin{titlepage}
\begin{center}
\vspace*{3cm}
\huge{\textbf{AHS Micro-Air Vehicle Challenge}}\par
\vspace{1cm}
\large{Project Sponsored By:}\par
\vspace{3mm}
\normalsize{Nancy Squires, Ph.D.}\par
\vspace{1cm}
\huge{\textbf{Problem Statement}}\par
\vspace{1cm}
\large{Written by:}\par
\vspace{3mm}
\normalsize{Justin Sherburne}\par
\vspace{25mm}
\large{\textbf{Abstract:}}\par
\vspace{4mm}
\normalsize{The AHS Micro-Air Vehicle challenge is a competiton hosted by the 
American Helicopter Society. This project is sponsored by Nancy Squires Ph.D., 
and will conclude in May 2018. The Micro-Air Vehicle project proposes to build 
an autonomous quadcopter capable of meeting the design requirements specified 
by the AHS competition. If the Oregon State University team qualifies to 
compete they will travel to Phoenix, AZ and participate in the competiton 
on May 14-17.}
\end{center}
\end{titlepage}

\setlength\parindent{4mm}
\section*{Problem Definition}\par
\hspace{4ex}Oregon State University will be competing in the American Helicopter 
Society's 2018 Micro-Air Vehicle Challenge. Our task is to design and build 
a vehicle to compete in the 2018 challenge located in Phoenix, Arizona on May 
14-17, 2018. This vehicle must meet the competition requirements set by The 
American Helicopter Society. Additionally, this vehicle must be capable of 
autonomous flight, and support an on-board video system. 

\section*{Problem Description}\par
\hspace{4ex}The Micro-Air Vehicle challenge is an annual collegiate competition
hosted by The American Helicopter Society. The competition in conducted by the 
AHS International's Unmanned VTOL Aircraft and Rotorcraft Committee. Competitors 
are tasked with building a vehicle that is capable of vertical takeoff and 
landing. This vehicle may compete in either a manual or autonomous flight 
category. While the 2018 competition rules have not yet been announced, last 
year's competition included restrictions on size, weight, and movement. Student 
teams may receive awards based on innovative systems implemented on their 
vehicles. The competition takes place at Forum 74, an annual convention hosted 
by AHS International. The forum takes place May 14-17, and the competition will 
take place on one of those days. The announcement of the competition date, and 
rules will be announced mid-October. 

\section*{Proposed Solution}\par
\hspace{4ex}The Oregon State University Micro-Air Vehicle team will consist of 
three mechanical engineering, three electrical engineering, and three computer 
science students. Initially the team will design and build a quadcopter that 
can be flown manually. When that goal is met, we will begin pursuing limited
autonomous flight. This limited autonomous flight should be able to navigate, 
without human interaction, on a pre-defined flight path. If that goal is 
achieved, then work will begin on a fully-autonomous flight function that 
allows us to meet the competition objectives without any human interaction. 
According to last year's rules, the vehicle must be under 500 grams, and less 
than 17.7 inches in any dimension. The vehicle must also have some sort of 
on-board camera system that is capable of streaming video to a computer. This 
video stream must be capable of 24fps, with a latency less than 150ms. 
According to last years rules, paper design proposals must be submitted by mid-
January, and a final video evidence of competition readiness much be submitted 
in March (exact dates to be determined).

\section*{Performance Metrics}\par
\hspace{4ex}Project completion will be judged by the following performance 
metrics:
\begin{enumerate}
\item Vehicle Completion:
\begin{itemize}
\item Can the vehicle fly under its own power?
\item Does it weigh less than 500g?
\item Is the vehicle sorter than 17.7 inches in any dimension?
\item Is the vehicle able to maintain a stable flight path?
\end{itemize}
\item Autonomous Completion:
\begin{itemize}
\item Can the vehicle stream video to a remote station?
\item Can the vehicle be controlled remotely from a computer?
\item Can the vehicle follow a specific flight plan without human interaction?
\item Can the vehicle locate targets and choose a flight plan without human
interaction?
\end{itemize}
\item Competiton Completion:
\begin{itemize}
\item The team must submit an intent to compete in the 2018 AHS Mirco-Air Vehicle 
Challenge.
\item The team must submit a design proposal for competition before the mid-
January cut-off date. 
\item The team must demonstrate competition readiness by meeting the vehicle and 
autonomous requirements prior to the competition date, and submit video evidence 
to the competition organizers for final selection.
\item If selected to compete in the 2018 challenge, the team must travel to 
Phoenix to participate in the challenge. 
\end{itemize}
\end{enumerate}
\end{document}




